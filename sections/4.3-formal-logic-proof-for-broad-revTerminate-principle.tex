\section{Formal Logic Proof for the Broad RevTerminate Principle}

\subsection{ Step }
\begin{code}%
    \>[2]\AgdaFunction{Finite-Reachable-State-Termination} %
    \AgdaSymbol{:} %
    \AgdaSymbol{∀} %
    \AgdaSymbol{\{}\AgdaBound{N} %
    \AgdaBound{st₀}\AgdaSymbol{\}}\<%
    \\
    \>[2][@{}l@{\AgdaIndent{0}}]%
    \>[4]\AgdaSymbol{→} %
    \AgdaSymbol{(}\AgdaBound{St-Fin} %
    \AgdaSymbol{:} %
    \AgdaFunction{∃[} %
    \AgdaBound{m} %
    \AgdaFunction{]} %
    \AgdaFunction{∃[} %
    \AgdaBound{stₘ} %
    \AgdaFunction{]} %
    \AgdaSymbol{(}\AgdaBound{st₀} %
    \AgdaOperator{\AgdaDatatype{↦[}} %
    \AgdaBound{m} %
    \AgdaOperator{\AgdaDatatype{]}} %
    \AgdaBound{stₘ}\AgdaSymbol{)} %
    \AgdaOperator{\AgdaFunction{⤖}} %
    \AgdaDatatype{Fin} %
    \AgdaBound{N}\AgdaSymbol{)}\<%
    \\
    %
    \>[4]\AgdaSymbol{→} %
    \AgdaSymbol{(}\AgdaBound{has-next} %
    \AgdaSymbol{:} %
    \AgdaSymbol{∀} %
    \AgdaSymbol{(}\AgdaBound{st} %
    \AgdaSymbol{:} %
    \AgdaField{State}\AgdaSymbol{)} %
    \AgdaSymbol{→} %
    \AgdaRecord{Dec} %
    \AgdaSymbol{(}\AgdaFunction{∃[} %
    \AgdaBound{st'} %
    \AgdaFunction{]} %
    \AgdaSymbol{(}\AgdaBound{st} %
    \AgdaOperator{\AgdaField{↦}} %
    \AgdaBound{st'}\AgdaSymbol{)))}\<%
    \\
    %
    \>[4]\AgdaSymbol{→} %
    \AgdaFunction{is-initial} %
    \AgdaBound{st₀}\<%
    \\
    %
    \>[4]\AgdaSymbol{→} %
    \AgdaFunction{∃[} %
    \AgdaBound{stₙ} %
    \AgdaFunction{]} %
    \AgdaSymbol{(}\AgdaBound{st₀} %
    \AgdaOperator{\AgdaDatatype{↦*}} %
    \AgdaBound{stₙ} %
    \AgdaOperator{\AgdaFunction{×}} %
    \AgdaFunction{is-stuck} %
    \AgdaBound{stₙ}\AgdaSymbol{)}\<%
    \\
\end{code}

Similar to informal logic proof, we prove the principle using the following steps:
\begin{enumerate}[1.]
    \item Starting from the initial state, continue steping into the next state.
    \item If the next state doesn't exist, then the machine terminates.
    \item After walking $\mathbb{N}$ steps, demonstrate a contradiction using the Piegonhole Principle and Mapping Rules.
    \item At this point, we have proven the case within $\mathbb{N}$ steps and shown that it's impossible for the case to extend beyond $\mathbb{N}$ steps.
\end{enumerate}

\subsection{ Countdown Rule }
Here is the Countdown Rule for Broad RevTerminate Principle.
Same as the narrow version, we introduce the variable ``m'' to represent stepping in, and create a corresponding variable ``countdown'' to ensure the proof terminates.
\begin{code}%
\>[2]\AgdaFunction{is-initial} %
\AgdaSymbol{:} %
\AgdaField{State} %
\AgdaSymbol{→} %
\AgdaPrimitive{Set} %
\AgdaSymbol{\AgdaUnderscore{}}\<%
\\
%
\>[2]\AgdaFunction{is-initial} %
\AgdaBound{st} %
\AgdaSymbol{=} %
\AgdaFunction{∄[} %
\AgdaBound{st'} %
\AgdaFunction{]} %
\AgdaSymbol{(}\AgdaBound{st'} %
\AgdaOperator{\AgdaField{↦}} %
\AgdaBound{st}\AgdaSymbol{)}\<%
\\
\>[0]\<%
\end{code}

\subsection{ Case at N }
Here is the statement of Broad revTermination after $\mathbb{N}$ steps, where we have to prove the case is impossible:
\begin{code}%
    \\
    %
    \>[4]\AgdaPostulate{Finite-State-Termination-At-N} %
    \AgdaSymbol{:} %
    \AgdaSymbol{∀} %
    \AgdaSymbol{\{}\AgdaBound{N} %
    \AgdaBound{st₀}\AgdaSymbol{\}}\<%
    \\
    \>[4][@{}l@{\AgdaIndent{0}}]%
    \>[8]\AgdaSymbol{→} %
    \AgdaField{State} %
    \AgdaOperator{\AgdaFunction{⤖}} %
    \AgdaDatatype{Fin} %
    \AgdaBound{N}\<%
    \\
    %
    \>[8]\AgdaSymbol{→} %
    \AgdaFunction{is-initial} %
    \AgdaBound{st₀}\<%
    \\
    %
    \>[8]\AgdaSymbol{→} %
    \AgdaFunction{∃[} %
    \AgdaBound{stₙ} %
    \AgdaFunction{]} %
    \AgdaSymbol{(}\AgdaBound{st₀} %
    \AgdaOperator{\AgdaDatatype{↦[}} %
    \AgdaBound{N} %
    \AgdaOperator{\AgdaDatatype{]}} %
    \AgdaBound{stₙ}\AgdaSymbol{)} %
    \AgdaSymbol{→} %
    \AgdaFunction{⊥}\<%
    \>[0]\<%
\end{code}

The proof of ``termination at the n-th state'' can be simplified into the following two steps:
\begin{enumerate}[1.]
    \item Assume that n+1-th state exists.  With the Piegonhole principle, we can map n states to n+1 states, thus finding two identical reachable states.
    \item In fact, we don't need No-Repeat Principle here because the definition of a reachable state itself is sufficient to get generate a contradiction.
\end{enumerate}.

We will first discuss the mapping rules for the Broad RevTerminate Principle.

\subsection{ Mapping Rules }
In our proof of termination as the n-th case, a critical component is the mapping rules.
We establish:
\begin{enumerate}[1.]
    \item An injection from Step to reachable state.  Note that the domain of Step is ranges from 0 to $\mathbb{N}$ (inclusive).
    \item A bijection between reachable state and Index, due to the limited number of States, resulting in a total of $\mathbb{N}$ indices.
\end{enumerate}.
Here is a diagram illustrating these relationships:
\vspace{1em}

\usetikzlibrary{graphs, positioning, quotes, shapes.geometric}

\begin{document}
\begin{tikzpicture}[node distance=10pt]
    \node[draw]                         (Step 0)  {Step 0};
    \node[draw, below=30pt of Step 0]   (Step 1)  {Step 1};
    \node[draw, below=30pt of Step 1]   (Step 2)  {Step 2};
    \node[draw, below=30pt of Step 2]   (Step 3)  {Step 3};
    \node[draw, below=30pt of Step 3]   (Step n)  {Step n};
    
    \node[draw, right=30pt of Step 0]   (State 0)  {State 0};
    \node[draw, right=30pt of Step 1]   (State 1)  {State 1};
    \node[draw, right=30pt of Step 2]   (State 2)  {State 2};
    \node[draw, right=30pt of Step 3]   (State 3)  {State 3};
    \node[draw, right=30pt of Step n]   (State n)  {State n};

    \node[draw, right=30pt of State 0]   (Index 0)  {Index 0};
    \node[draw, right=30pt of State 1]   (Index 1)  {Index 1};
    \node[draw, right=30pt of State 2]   (Index 2)  {Index 2};
    \node[draw, right=30pt of State 3]   (Index 3)  {Index 3};
    \node[draw, right=30pt of State n]   (Index n)  {  ?  };

    \graph{
        (Step 0) -> (State 0) -> (Index 1);
        (Step 1) -> (State 1) -> (Index 3);
        (Step 2) -> (State 2) -> (Index 0);
        (Step 3) -> (State 3) -> (Index 2);
        (Step n) -> (State n) -> (Index n);
    };
\end{tikzpicture}

Note again that there are only $\mathbb{N}$ indices in the relationships.
With this relationships, we can clearly describe the Termination proof at the n-th state:
\begin{enumerate}[1.]
    \item Initially, we have n+1 states (from $st_{0}$ to $st_{n}$).
    \item With the injection from Step to Index, using the Piegonhole Principle, we identify two different steps that map to the identical index.
    \item With the bijection from Index back to State, the same indices must map to the same reachable states.
\end{enumerate}.

\subsection{ Piegonhole principle }
The definition of Piegonhole principle in Agda is shown below:

\begin{code}%
    \\
    %
    \>[4]\AgdaPostulate{pigeonhole} %
    \AgdaSymbol{:} %
    \AgdaSymbol{∀} %
    \AgdaBound{N} %
    \AgdaSymbol{→} %
    \AgdaSymbol{(}\AgdaBound{f} %
    \AgdaSymbol{:} %
    \AgdaDatatype{ℕ} %
    \AgdaSymbol{→} %
    \AgdaDatatype{ℕ}\AgdaSymbol{)}\<%
    \\
    \>[4][@{}l@{\AgdaIndent{0}}]%
    \>[11]\AgdaSymbol{→} %
    \AgdaSymbol{(∀} %
    \AgdaBound{n} %
    \AgdaSymbol{→} %
    \AgdaBound{n} %
    \AgdaOperator{\AgdaDatatype{≤}} %
    \AgdaBound{N} %
    \AgdaSymbol{→} %
    \AgdaBound{f} %
    \AgdaBound{n} %
    \AgdaOperator{\AgdaFunction{<}} %
    \AgdaBound{N}\AgdaSymbol{)}\<%
    \\
    %
    \>[11]\AgdaSymbol{→} %
    \AgdaFunction{∃[} %
    \AgdaBound{m} %
    \AgdaFunction{]} %
    \AgdaFunction{∃[} %
    \AgdaBound{n} %
    \AgdaFunction{]} %
    \AgdaSymbol{(}\AgdaBound{m} %
    \AgdaOperator{\AgdaFunction{<}} %
    \AgdaBound{n} %
    \AgdaOperator{\AgdaFunction{×}} %
    \AgdaBound{n} %
    \AgdaOperator{\AgdaDatatype{≤}} %
    \AgdaBound{N} %
    \AgdaOperator{\AgdaFunction{×}} %
    \AgdaBound{f} %
    \AgdaBound{m} %
    \AgdaOperator{\AgdaDatatype{≡}} %
    \AgdaBound{f} %
    \AgdaBound{n}\AgdaSymbol{)}\<%
    \>[0]\<%
\end{code}

This principle requires two key inputs:
\begin{enumerate}[1.]
    \item A injection from natural number to another natural number. In this case, it is the injection from Step to Index. 
    \item All mapped number should be less than $\mathbb{N}$.  The proof is inherent in the definition of Fin.  That is, a natrual number value of Fin $\mathbb{N}$ is always less than $\mathbb{N}$.
\end{enumerate}.

Using these inputs, we can derive two different steps that map to the same index.

\subsection{ Contradiction }
Here, we do not require the No-Repeat Principle.  
The Piegonhole Principle indicates that two different steps correspond to the same index; through bijection, this same index maps back to two identical reachable states.
The definition of reachable states inherently includes the step in the trace, allowing us to straightforwardly prove that the steps are the same using the following codes:
\begin{code}
    %\\
    %
    \>[4]\AgdaPostulate{Finite-Reachable-State-Termination-CountDown} %
    \AgdaSymbol{:} %
    \AgdaSymbol{∀} %
    \AgdaSymbol{\{}\AgdaBound{N} %
    \AgdaBound{st₀}\AgdaSymbol{\}}\<%
    \\
    \>[4][@{}l@{\AgdaIndent{0}}]%
    \>[6]\AgdaSymbol{→} %
    \AgdaSymbol{(}\AgdaBound{St-Fin} %
    \AgdaSymbol{:} %
    \AgdaFunction{∃[} %
    \AgdaBound{m} %
    \AgdaFunction{]} %
    \AgdaFunction{∃[} %
    \AgdaBound{stₘ} %
    \AgdaFunction{]} %
    \AgdaSymbol{(}\AgdaBound{st₀} %
    \AgdaOperator{\AgdaDatatype{↦[}} %
    \AgdaBound{m} %
    \AgdaOperator{\AgdaDatatype{]}} %
    \AgdaBound{stₘ}\AgdaSymbol{)} %
    \AgdaOperator{\AgdaFunction{⤖}} %
    \AgdaDatatype{Fin} %
    \AgdaBound{N}\AgdaSymbol{)}\<%
    \\
    %
    \>[6]\AgdaSymbol{→} %
    \AgdaSymbol{(}\AgdaBound{has-next} %
    \AgdaSymbol{:} %
    \AgdaSymbol{∀} %
    \AgdaSymbol{(}\AgdaBound{st} %
    \AgdaSymbol{:} %
    \AgdaField{State}\AgdaSymbol{)} %
    \AgdaSymbol{→} %
    \AgdaRecord{Dec} %
    \AgdaSymbol{(}\AgdaFunction{∃[} %
    \AgdaBound{st'} %
    \AgdaFunction{]} %
    \AgdaSymbol{(}\AgdaBound{st} %
    \AgdaOperator{\AgdaField{↦}} %
    \AgdaBound{st'}\AgdaSymbol{)))}\<%
    \\
    %
    \>[6]\AgdaSymbol{→} %
    \AgdaFunction{is-initial} %
    \AgdaBound{st₀}\<%
    \\
    %
    \>[6]\AgdaSymbol{→} %
    \AgdaSymbol{∀} %
    \AgdaBound{cd} %
    \AgdaBound{m}%
    \>[16]\AgdaBound{stₘ} %
    \AgdaSymbol{→} %
    \AgdaBound{cd} %
    \AgdaOperator{\AgdaPrimitive{+}} %
    \AgdaBound{m} %
    \AgdaOperator{\AgdaDatatype{≡}} %
    \AgdaBound{N} %
    \AgdaSymbol{→} %
    \AgdaBound{st₀} %
    \AgdaOperator{\AgdaDatatype{↦[}} %
    \AgdaBound{m} %
    \AgdaOperator{\AgdaDatatype{]}} %
    \AgdaBound{stₘ}\<%
    \\
    %
    \>[6]\AgdaSymbol{→} %
    \AgdaFunction{∃[} %
    \AgdaBound{stₙ} %
    \AgdaFunction{]} %
    \AgdaSymbol{(}\AgdaBound{st₀} %
    \AgdaOperator{\AgdaDatatype{↦*}} %
    \AgdaBound{stₙ} %
    \AgdaOperator{\AgdaFunction{×}} %
    \AgdaFunction{is-stuck} %
    \AgdaBound{stₙ}\AgdaSymbol{)}\<%
    \>[0]\<%
\end{code}

Therefore, we arrive at the conclusion that the two steps are simultaneously the same and different, presenting an obvious contradiction.

With this, we have addressed the remaining elements of the Broad RevTerminate Principle, thereby completing the overall proof.
