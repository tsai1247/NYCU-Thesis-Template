\section{Formal Logic Proof for Broad RevTerminate Principle}

\subsection{ Step }
\begin{code}%
    \>[2]\AgdaFunction{Finite-Reachable-State-Termination} %
    \AgdaSymbol{:} %
    \AgdaSymbol{∀} %
    \AgdaSymbol{\{}\AgdaBound{N} %
    \AgdaBound{st₀}\AgdaSymbol{\}}\<%
    \\
    \>[2][@{}l@{\AgdaIndent{0}}]%
    \>[4]\AgdaSymbol{→} %
    \AgdaSymbol{(}\AgdaBound{St-Fin} %
    \AgdaSymbol{:} %
    \AgdaFunction{∃[} %
    \AgdaBound{m} %
    \AgdaFunction{]} %
    \AgdaFunction{∃[} %
    \AgdaBound{stₘ} %
    \AgdaFunction{]} %
    \AgdaSymbol{(}\AgdaBound{st₀} %
    \AgdaOperator{\AgdaDatatype{↦[}} %
    \AgdaBound{m} %
    \AgdaOperator{\AgdaDatatype{]}} %
    \AgdaBound{stₘ}\AgdaSymbol{)} %
    \AgdaOperator{\AgdaFunction{⤖}} %
    \AgdaDatatype{Fin} %
    \AgdaBound{N}\AgdaSymbol{)}\<%
    \\
    %
    \>[4]\AgdaSymbol{→} %
    \AgdaSymbol{(}\AgdaBound{has-next} %
    \AgdaSymbol{:} %
    \AgdaSymbol{∀} %
    \AgdaSymbol{(}\AgdaBound{st} %
    \AgdaSymbol{:} %
    \AgdaField{State}\AgdaSymbol{)} %
    \AgdaSymbol{→} %
    \AgdaRecord{Dec} %
    \AgdaSymbol{(}\AgdaFunction{∃[} %
    \AgdaBound{st'} %
    \AgdaFunction{]} %
    \AgdaSymbol{(}\AgdaBound{st} %
    \AgdaOperator{\AgdaField{↦}} %
    \AgdaBound{st'}\AgdaSymbol{)))}\<%
    \\
    %
    \>[4]\AgdaSymbol{→} %
    \AgdaFunction{is-initial} %
    \AgdaBound{st₀}\<%
    \\
    %
    \>[4]\AgdaSymbol{→} %
    \AgdaFunction{∃[} %
    \AgdaBound{stₙ} %
    \AgdaFunction{]} %
    \AgdaSymbol{(}\AgdaBound{st₀} %
    \AgdaOperator{\AgdaDatatype{↦*}} %
    \AgdaBound{stₙ} %
    \AgdaOperator{\AgdaFunction{×}} %
    \AgdaFunction{is-stuck} %
    \AgdaBound{stₙ}\AgdaSymbol{)}\<%
    \\
\end{code}

Similar to informal logic proof, we proof the principle in such steps below:
\begin{enumerate}[1.]
    \item Starting from initial state, keep steping into the next state.
    \item If the next state doesn't exist, then it terminate.
    \item Upon walked N steps, show contradiction by Piegonhole Principle and Mapping Rules.
    \item 至此,we have proved the case in N steps and it's impossible for the case over N steps.
\end{enumerate}

\subsection{ Countdown Rule }
Here is the Countdown Rule for Broad RevTerminate Principle.
Same as the narrow one, we add the variable ``m'' to 呈現 a step-in, and make a 相對的 variable ``countdown'' to terminate the proof.
\begin{code}%
\>[2]\AgdaFunction{is-initial} %
\AgdaSymbol{:} %
\AgdaField{State} %
\AgdaSymbol{→} %
\AgdaPrimitive{Set} %
\AgdaSymbol{\AgdaUnderscore{}}\<%
\\
%
\>[2]\AgdaFunction{is-initial} %
\AgdaBound{st} %
\AgdaSymbol{=} %
\AgdaFunction{∄[} %
\AgdaBound{st'} %
\AgdaFunction{]} %
\AgdaSymbol{(}\AgdaBound{st'} %
\AgdaOperator{\AgdaField{↦}} %
\AgdaBound{st}\AgdaSymbol{)}\<%
\\
\>[0]\<%
\end{code}

\subsection{ The case at N }
Here is the statement of Broad revTermination after n steps, we have to proof the case is impossible:
\begin{code}%
    \\
    %
    \>[4]\AgdaPostulate{Finite-State-Termination-At-N} %
    \AgdaSymbol{:} %
    \AgdaSymbol{∀} %
    \AgdaSymbol{\{}\AgdaBound{N} %
    \AgdaBound{st₀}\AgdaSymbol{\}}\<%
    \\
    \>[4][@{}l@{\AgdaIndent{0}}]%
    \>[8]\AgdaSymbol{→} %
    \AgdaField{State} %
    \AgdaOperator{\AgdaFunction{⤖}} %
    \AgdaDatatype{Fin} %
    \AgdaBound{N}\<%
    \\
    %
    \>[8]\AgdaSymbol{→} %
    \AgdaFunction{is-initial} %
    \AgdaBound{st₀}\<%
    \\
    %
    \>[8]\AgdaSymbol{→} %
    \AgdaFunction{∃[} %
    \AgdaBound{stₙ} %
    \AgdaFunction{]} %
    \AgdaSymbol{(}\AgdaBound{st₀} %
    \AgdaOperator{\AgdaDatatype{↦[}} %
    \AgdaBound{N} %
    \AgdaOperator{\AgdaDatatype{]}} %
    \AgdaBound{stₙ}\AgdaSymbol{)} %
    \AgdaSymbol{→} %
    \AgdaFunction{⊥}\<%
    \>[0]\<%
\end{code}

The proof of 「當抵達 n-th state 會終止」 can be simply分為 the two steps below:
\begin{enumerate}[1.]
    \item 假定n+1-th state存在, with Piegonhole principle ,可以在 n 個 state 映射到 n + 1 個state,而找到重複經過的兩個相同 reachable state
    \item In fact, we don't need no-repeat principle here because the definition of reachable state is enough to get contradiction.
\end{enumerate}.

We discussing mapping rules for broad revTerminate principle firstly.

\subsection{ mapping rules }
In our proof of n-th Termination case, a critical part is the mapping rules.
We have,
\begin{enumerate}[1.]
    \item an injection from Step to reachable state.  Note that the 定義域 of step is from 0 to N(included both sides)
    \item bijection between reachable state and Index. because of the limitation of States, we should have n indexes in total.
\end{enumerate}.
Here is an 示意圖 for the relations
\vspace{1em}

\usetikzlibrary{graphs, positioning, quotes, shapes.geometric}

\begin{document}
\begin{tikzpicture}[node distance=10pt]
    \node[draw]                         (Step 0)  {Step 0};
    \node[draw, below=30pt of Step 0]   (Step 1)  {Step 1};
    \node[draw, below=30pt of Step 1]   (Step 2)  {Step 2};
    \node[draw, below=30pt of Step 2]   (Step 3)  {Step 3};
    \node[draw, below=30pt of Step 3]   (Step n)  {Step n};
    
    \node[draw, right=30pt of Step 0]   (State 0)  {State 0};
    \node[draw, right=30pt of Step 1]   (State 1)  {State 1};
    \node[draw, right=30pt of Step 2]   (State 2)  {State 2};
    \node[draw, right=30pt of Step 3]   (State 3)  {State 3};
    \node[draw, right=30pt of Step n]   (State n)  {State n};

    \node[draw, right=30pt of State 0]   (Index 0)  {Index 0};
    \node[draw, right=30pt of State 1]   (Index 1)  {Index 1};
    \node[draw, right=30pt of State 2]   (Index 2)  {Index 2};
    \node[draw, right=30pt of State 3]   (Index 3)  {Index 3};
    \node[draw, right=30pt of State n]   (Index n)  {  ?  };

    \graph{
        (Step 0) -> (State 0) -> (Index 1);
        (Step 1) -> (State 1) -> (Index 3);
        (Step 2) -> (State 2) -> (Index 0);
        (Step 3) -> (State 3) -> (Index 2);
        (Step n) -> (State n) -> (Index n);
    };
\end{tikzpicture}

Note again that there are only n indexes in the relations.
With the relation, we can 更清楚地描述 the Termination proof at n-th state:
\begin{enumerate}[1.]
    \item first of all, we have n+1 states (from $st_{0}$ to $st_{n}$)
    \item with injection from Step to index, we get two different steps mapping to same index using Piegonhole Principle
    \item with bijection from index back to State, the same indexes should be mapped to same reachable states.
\end{enumerate}.

\subsection{ Piegonhole principle }
The definition of Piegonhole principle in agda is shown below:

\begin{code}%
    \\
    %
    \>[4]\AgdaPostulate{pigeonhole} %
    \AgdaSymbol{:} %
    \AgdaSymbol{∀} %
    \AgdaBound{N} %
    \AgdaSymbol{→} %
    \AgdaSymbol{(}\AgdaBound{f} %
    \AgdaSymbol{:} %
    \AgdaDatatype{ℕ} %
    \AgdaSymbol{→} %
    \AgdaDatatype{ℕ}\AgdaSymbol{)}\<%
    \\
    \>[4][@{}l@{\AgdaIndent{0}}]%
    \>[11]\AgdaSymbol{→} %
    \AgdaSymbol{(∀} %
    \AgdaBound{n} %
    \AgdaSymbol{→} %
    \AgdaBound{n} %
    \AgdaOperator{\AgdaDatatype{≤}} %
    \AgdaBound{N} %
    \AgdaSymbol{→} %
    \AgdaBound{f} %
    \AgdaBound{n} %
    \AgdaOperator{\AgdaFunction{<}} %
    \AgdaBound{N}\AgdaSymbol{)}\<%
    \\
    %
    \>[11]\AgdaSymbol{→} %
    \AgdaFunction{∃[} %
    \AgdaBound{m} %
    \AgdaFunction{]} %
    \AgdaFunction{∃[} %
    \AgdaBound{n} %
    \AgdaFunction{]} %
    \AgdaSymbol{(}\AgdaBound{m} %
    \AgdaOperator{\AgdaFunction{<}} %
    \AgdaBound{n} %
    \AgdaOperator{\AgdaFunction{×}} %
    \AgdaBound{n} %
    \AgdaOperator{\AgdaDatatype{≤}} %
    \AgdaBound{N} %
    \AgdaOperator{\AgdaFunction{×}} %
    \AgdaBound{f} %
    \AgdaBound{m} %
    \AgdaOperator{\AgdaDatatype{≡}} %
    \AgdaBound{f} %
    \AgdaBound{n}\AgdaSymbol{)}\<%
    \>[0]\<%
\end{code}

The two necessary inputs are 
\begin{enumerate}[1.]
    \item N to N function. In this case, it is the injection from Step to Index. 
    \item all mapped N should be less than N.  The proof is 隱含在 Fin的定義裡面.  That is, a natrual number value of Fin N is always less than N.
\end{enumerate}.

And we could get two different steps which will injection to the same index.

\subsection{ contradiction }
Here, we don't need the No-Repeat Principle.  
The piegonhole principle give us two different step and the same index, and we use bijection to make the same index back to two same reachable states.
the definition of reachable states 恰好 contains the step for the trace, we can simply proof the step is same by the following codes:
\begin{code}
    %\\
    %
    \>[4]\AgdaPostulate{Finite-Reachable-State-Termination-CountDown} %
    \AgdaSymbol{:} %
    \AgdaSymbol{∀} %
    \AgdaSymbol{\{}\AgdaBound{N} %
    \AgdaBound{st₀}\AgdaSymbol{\}}\<%
    \\
    \>[4][@{}l@{\AgdaIndent{0}}]%
    \>[6]\AgdaSymbol{→} %
    \AgdaSymbol{(}\AgdaBound{St-Fin} %
    \AgdaSymbol{:} %
    \AgdaFunction{∃[} %
    \AgdaBound{m} %
    \AgdaFunction{]} %
    \AgdaFunction{∃[} %
    \AgdaBound{stₘ} %
    \AgdaFunction{]} %
    \AgdaSymbol{(}\AgdaBound{st₀} %
    \AgdaOperator{\AgdaDatatype{↦[}} %
    \AgdaBound{m} %
    \AgdaOperator{\AgdaDatatype{]}} %
    \AgdaBound{stₘ}\AgdaSymbol{)} %
    \AgdaOperator{\AgdaFunction{⤖}} %
    \AgdaDatatype{Fin} %
    \AgdaBound{N}\AgdaSymbol{)}\<%
    \\
    %
    \>[6]\AgdaSymbol{→} %
    \AgdaSymbol{(}\AgdaBound{has-next} %
    \AgdaSymbol{:} %
    \AgdaSymbol{∀} %
    \AgdaSymbol{(}\AgdaBound{st} %
    \AgdaSymbol{:} %
    \AgdaField{State}\AgdaSymbol{)} %
    \AgdaSymbol{→} %
    \AgdaRecord{Dec} %
    \AgdaSymbol{(}\AgdaFunction{∃[} %
    \AgdaBound{st'} %
    \AgdaFunction{]} %
    \AgdaSymbol{(}\AgdaBound{st} %
    \AgdaOperator{\AgdaField{↦}} %
    \AgdaBound{st'}\AgdaSymbol{)))}\<%
    \\
    %
    \>[6]\AgdaSymbol{→} %
    \AgdaFunction{is-initial} %
    \AgdaBound{st₀}\<%
    \\
    %
    \>[6]\AgdaSymbol{→} %
    \AgdaSymbol{∀} %
    \AgdaBound{cd} %
    \AgdaBound{m}%
    \>[16]\AgdaBound{stₘ} %
    \AgdaSymbol{→} %
    \AgdaBound{cd} %
    \AgdaOperator{\AgdaPrimitive{+}} %
    \AgdaBound{m} %
    \AgdaOperator{\AgdaDatatype{≡}} %
    \AgdaBound{N} %
    \AgdaSymbol{→} %
    \AgdaBound{st₀} %
    \AgdaOperator{\AgdaDatatype{↦[}} %
    \AgdaBound{m} %
    \AgdaOperator{\AgdaDatatype{]}} %
    \AgdaBound{stₘ}\<%
    \\
    %
    \>[6]\AgdaSymbol{→} %
    \AgdaFunction{∃[} %
    \AgdaBound{stₙ} %
    \AgdaFunction{]} %
    \AgdaSymbol{(}\AgdaBound{st₀} %
    \AgdaOperator{\AgdaDatatype{↦*}} %
    \AgdaBound{stₙ} %
    \AgdaOperator{\AgdaFunction{×}} %
    \AgdaFunction{is-stuck} %
    \AgdaBound{stₙ}\AgdaSymbol{)}\<%
    \>[0]\<%
\end{code}

So, we have the 結論 of the two steps are both same and different, which makes aobvious contradiction.

Until now, we have 補足了 the 剩餘的部分 of Broad revTermination Principle, 從而補足了整體的證明。