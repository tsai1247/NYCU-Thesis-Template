\section{ informal Logic Proof }

We also use Mathematical Induction to proof the statement.  Here is the 流程

\begin{enumerate}[1.]
    \item given an initial state $st_{0}$
    \item if there is no step from $st_{0}$, then it terminate; otherwise, we reach $st_{1}$
    \item keep 推進 the trace, until reach $st_{n}$.
    \item When we reach $st_{n}$, we have traversed all n reachable states.
    \item if $st_{n}$ not terminate, consider the next state of $st_{n}$, the piegonhole principle says that there are two states will be same, but there are in different steps from $st_{0}$ obviously.
    \item The statement will occurs a contradiction with No-repeat principle.
\end{enumerate}.

Similar to the narrow reversible termination, however, the bijection 從 ``State to index'' 變成了 ``State to reachable state index''.
In fact, in informal logic proof, 除了 the bijection 被替換了, all processes are the same.
