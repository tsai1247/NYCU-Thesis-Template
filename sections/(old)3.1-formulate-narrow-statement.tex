\section{Formulate Statement}

Before we can construct the Termination Statement, it is necessary to define certain relationships between states.

\begin{code}%
\>[2]\AgdaFunction{is-initial} %
\AgdaSymbol{:} %
\AgdaField{State} %
\AgdaSymbol{→} %
\AgdaPrimitive{Set} %
\AgdaSymbol{\AgdaUnderscore{}}\<%
\\
%
\>[2]\AgdaFunction{is-initial} %
\AgdaBound{st} %
\AgdaSymbol{=} %
\AgdaFunction{∄[} %
\AgdaBound{st'} %
\AgdaFunction{]} %
\AgdaSymbol{(}\AgdaBound{st'} %
\AgdaOperator{\AgdaField{↦}} %
\AgdaBound{st}\AgdaSymbol{)}\<%
\\
\>[0]\<%
\end{code}
The is-initial code segment provides a proof that a given state has no preceding states.

\begin{code}%
\>[2]\AgdaFunction{is-stuck} %
\AgdaSymbol{:} %
\AgdaField{State} %
\AgdaSymbol{→} %
\AgdaPrimitive{Set} %
\AgdaSymbol{\AgdaUnderscore{}}\<%
\\
%
\>[2]\AgdaFunction{is-stuck} %
\AgdaBound{st} %
\AgdaSymbol{=} %
\AgdaFunction{∄[} %
\AgdaBound{st'} %
\AgdaFunction{]} %
\AgdaSymbol{(}\AgdaBound{st} %
\AgdaOperator{\AgdaField{↦}} %
\AgdaBound{st'}\AgdaSymbol{)}\<%
\\
\>[0]\<%
\end{code}
The is-stuck code segment offers a proof that a given state has no succeeding states.

\begin{code}%
\>[2]\AgdaKeyword{data} %
\AgdaOperator{\AgdaDatatype{\underline{\hspace{0.5em}}↦*\underline{\hspace{0.5em}}}} %
\AgdaSymbol{:} %
\AgdaField{State} %
\AgdaSymbol{→} %
\AgdaField{State} %
\AgdaSymbol{→} %
\AgdaPrimitive{Set} %
\AgdaSymbol{(}\AgdaPrimitive{suc} %
\AgdaBound{ℓ}\AgdaSymbol{)} %
\AgdaKeyword{where}\<%
\\
\>[2][@{}l@{\AgdaIndent{0}}]%
\>[4]\AgdaInductiveConstructor{◾} %
\AgdaSymbol{:} %    
\AgdaSymbol{\{}\AgdaBound{st} %
\AgdaSymbol{:} %
\AgdaField{State}\AgdaSymbol{\}} %
\AgdaSymbol{→} %
\AgdaBound{st} %
\AgdaOperator{\AgdaDatatype{↦*}} %
\AgdaBound{st}\<%
\\
%
\>[4]\AgdaOperator{\AgdaInductiveConstructor{\underline{\hspace{0.5em}}∷\underline{\hspace{0.5em}}}} %
\AgdaSymbol{:} %
\AgdaSymbol{\{}\AgdaBound{st₁} %
\AgdaBound{st₂} %
\AgdaBound{st₃} %
\AgdaSymbol{:} %
\AgdaField{State}\AgdaSymbol{\}} %
\AgdaSymbol{→} %
\AgdaBound{st₁} %
\AgdaOperator{\AgdaField{↦}} %
\AgdaBound{st₂} %
\AgdaSymbol{→} %
\AgdaBound{st₂} %
\AgdaOperator{\AgdaDatatype{↦*}} %
\AgdaBound{st₃} %
\AgdaSymbol{→} %
\AgdaBound{st₁} %
\AgdaOperator{\AgdaDatatype{↦*}} %
\AgdaBound{st₃}\<%
\\
\>[0]\<%
\end{code}
The ↦* symbol code segment establishes a trace between two specified states, demonstrating that one state can be reached from the other through a finite number of transitions.

\begin{code}%
\>[2]\AgdaKeyword{data} %
\AgdaOperator{\AgdaDatatype{\underline{\hspace{0.5em}}↦[\underline{\hspace{0.5em}}]\underline{\hspace{0.5em}}}} %
\AgdaSymbol{:} %
\AgdaField{State} %
\AgdaSymbol{→} %
\AgdaDatatype{ℕ} %
\AgdaSymbol{→} %
\AgdaField{State} %
\AgdaSymbol{→} %
\AgdaPrimitive{Set} %
\AgdaSymbol{(}\AgdaPrimitive{L.suc} %
\AgdaBound{ℓ}\AgdaSymbol{)} %
\AgdaKeyword{where}\<%
\\
\>[2][@{}l@{\AgdaIndent{0}}]%
\>[4]\AgdaInductiveConstructor{◾} %
\AgdaSymbol{:} %
\AgdaSymbol{∀} %
\AgdaSymbol{\{}\AgdaBound{st}\AgdaSymbol{\}} %
\AgdaSymbol{→} %
\AgdaBound{st} %
\AgdaOperator{\AgdaDatatype{↦[}} %
\AgdaNumber{0} %
\AgdaOperator{\AgdaDatatype{]}} %
\AgdaBound{st}\<%
\\
%
\>[4]\AgdaOperator{\AgdaInductiveConstructor{\underline{\hspace{0.5em}}∷\underline{\hspace{0.5em}}}} %
\AgdaSymbol{:} %
\AgdaSymbol{∀} %
\AgdaSymbol{\{}\AgdaBound{st₁} %
\AgdaBound{st₂} %
\AgdaBound{st₃} %
\AgdaBound{n}\AgdaSymbol{\}} %
\AgdaSymbol{→} %
\AgdaBound{st₁} %
\AgdaOperator{\AgdaField{↦}} %
\AgdaBound{st₂} %
\AgdaSymbol{→} %
\AgdaBound{st₂} %
\AgdaOperator{\AgdaDatatype{↦[}} %
\AgdaBound{n} %
\AgdaOperator{\AgdaDatatype{]}} %
\AgdaBound{st₃} %
\AgdaSymbol{→} %
\AgdaBound{st₁} %
\AgdaOperator{\AgdaDatatype{↦[}} %
\AgdaInductiveConstructor{suc} %
\AgdaBound{n} %
\AgdaOperator{\AgdaDatatype{]}} %
\AgdaBound{st₃}\<%
\\
\>[0]\<%
\end{code}
Similar to the ↦* code, the ↦[n] segment specifies a definite trace length n, establishing a fixed number of transitions from one state to another.

First of all, we describe in natural language the proof content expected for narrow reversible termination:
In a reversible machine m, if the number of states in the state set is finite, then each initial state should reach a stuck state after a finite number of transitions and terminate.

Here is the formal definition of a narrow reversible termination statement:

\begin{code}%
\>[2]\AgdaKeyword{postulate}\<%
\\
\>[2][@{}l@{\AgdaIndent{0}}]%
\>[4]\AgdaPostulate{Finite-State-Termination} %
\AgdaSymbol{:} %
\AgdaSymbol{∀} %
\AgdaSymbol{\{}\AgdaBound{N} %
\AgdaBound{st₀}\AgdaSymbol{\}}\<%
\\
\>[4][@{}l@{\AgdaIndent{0}}]%
\>[6]\AgdaSymbol{→} %
\AgdaSymbol{(∀} %
\AgdaSymbol{(}\AgdaBound{st} %
\AgdaSymbol{:} %
\AgdaField{State}\AgdaSymbol{)} %
\AgdaSymbol{→} %
\AgdaRecord{Dec} %
\AgdaSymbol{(}\AgdaFunction{∃[} %
\AgdaBound{st'} %
\AgdaFunction{]} %
\AgdaSymbol{(}\AgdaBound{st} %
\AgdaOperator{\AgdaField{↦}} %
\AgdaBound{st'}\AgdaSymbol{)))}\<%
\\
%
\>[6]\AgdaSymbol{→} %
\AgdaField{State} %
\AgdaOperator{\AgdaFunction{⤖}} %
\AgdaDatatype{Fin} %
\AgdaBound{N}\<%
\\
%
\>[6]\AgdaSymbol{→} %
\AgdaFunction{is-initial} %
\AgdaBound{st₀}\<%
\\
%
\>[6]\AgdaSymbol{→} %
\AgdaFunction{∃[} %
\AgdaBound{stₙ} %
\AgdaFunction{]} %
\AgdaSymbol{(}\AgdaBound{st₀} %
\AgdaOperator{\AgdaDatatype{↦*}} %
\AgdaBound{stₙ} %
\AgdaOperator{\AgdaFunction{×}} %
\AgdaFunction{is-stuck} %
\AgdaBound{stₙ}\AgdaSymbol{)}\<%
\end{code}

In the first two statements, we describe the necessary conditions for the termination of a reversible machine in narrow cases:

\begin{enumerate}[1.]
\item "For every state, the existence of a subsequent state is decidable." Imagine if the machine cannot determine whether a state can continue to progress; in such cases, the state would not be able to advance.
\item There is a bijection between the set of States and Fin N. This statement restricts the total number of states by establishing a correspondence with Fin N.
\end{enumerate}

And in the former 2 statement, we 描述了Termination這件事本身:

Given an initial state, we can determine a reachable stuck state, ensuring that the machine will eventually terminate.






