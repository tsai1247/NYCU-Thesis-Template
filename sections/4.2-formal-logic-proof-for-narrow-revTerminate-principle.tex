\section{Formal Logic Proof for Narrow RevTerminate Principle}

\subsection{ Step }
\begin{code}%
    \>[2]\AgdaFunction{Finite-State-Termination} %
    \AgdaSymbol{:} %
    \AgdaSymbol{∀} %
    \AgdaSymbol{\{}\AgdaBound{N} %
    \AgdaBound{st₀}\AgdaSymbol{\}}\<%
    \\
    \>[2][@{}l@{\AgdaIndent{0}}]%
    \>[4]\AgdaSymbol{→} %
    \AgdaField{State} %
    \AgdaOperator{\AgdaFunction{⤖}} %
    \AgdaDatatype{Fin} %
    \AgdaBound{N}\<%
    \\
    %
    \>[4]\AgdaSymbol{→} %
    \AgdaSymbol{(}\AgdaBound{has-next} %
    \AgdaSymbol{:} %
    \AgdaSymbol{∀} %
    \AgdaSymbol{(}\AgdaBound{st} %
    \AgdaSymbol{:} %
    \AgdaField{State}\AgdaSymbol{)} %
    \AgdaSymbol{→} %
    \AgdaRecord{Dec} %
    \AgdaSymbol{(}\AgdaFunction{∃[} %
    \AgdaBound{st'} %
    \AgdaFunction{]} %
    \AgdaSymbol{(}\AgdaBound{st} %
    \AgdaOperator{\AgdaField{↦}} %
    \AgdaBound{st'}\AgdaSymbol{)))}\<%
    \\
    %
    \>[4]\AgdaSymbol{→} %
    \AgdaFunction{is-initial} %
    \AgdaBound{st₀}\<%
    \\
    %
    \>[4]\AgdaSymbol{→} %
    \AgdaFunction{∃[} %
    \AgdaBound{stₙ} %
    \AgdaFunction{]} %
    \AgdaSymbol{(}\AgdaBound{st₀} %
    \AgdaOperator{\AgdaDatatype{↦*}} %
    \AgdaBound{stₙ} %
    \AgdaOperator{\AgdaFunction{×}} %
    \AgdaFunction{is-stuck} %
    \AgdaBound{stₙ}\AgdaSymbol{)}\<%
    \\
    \>[0]\<%
\end{code}

Similar to informal logic proof, we proof the principle in such steps below:
\begin{enumerate}[1.]
    \item Starting from initial state, keep steping into the next state.
    \item If the next state doesn't exist, then it terminate.
    \item Upon walked N steps, show contradiction by Piegonhole Principle and No-Repeat Principle.
    \item 至此,we have proved the case in N steps and it's impossible for the case over N steps.
\end{enumerate}

\subsection{ Countdown Rule }
To realize the step of ``keep walking into the next step'', we add a step m and 對應的 state $st_{m}$.

\begin{code}%
\>[2]\AgdaFunction{is-initial} %
\AgdaSymbol{:} %
\AgdaField{State} %
\AgdaSymbol{→} %
\AgdaPrimitive{Set} %
\AgdaSymbol{\AgdaUnderscore{}}\<%
\\
%
\>[2]\AgdaFunction{is-initial} %
\AgdaBound{st} %
\AgdaSymbol{=} %
\AgdaFunction{∄[} %
\AgdaBound{st'} %
\AgdaFunction{]} %
\AgdaSymbol{(}\AgdaBound{st'} %
\AgdaOperator{\AgdaField{↦}} %
\AgdaBound{st}\AgdaSymbol{)}\<%
\\
\>[0]\<%
\end{code}

When m is less than N, we try steping into the next state.  If there isn't a next state, we got the target stuck state.  Otherwise, we make a recursion with the case of m increasing.
Until m equals to N, we have to proof that the case is impossible.
\input{agdaCode/4.2-Narrow-RevTerminate-Principle-With-M-Proof}

For the 證明方式 above, agda will give an ``Termination checking failed'' error.
To make the proof terminate, we have to create a variable ``countdown'' in the statement.  The ``countdown'' should keep decreasing until it equals to zero.  And we proof the case zero at last.
The ``countdown'' 恰好與 m 相對.  We keep providing the ``cd + m $\equiv$ N.''  In this way, the proof will terminate at m exactly equals to N, 就如原先的預想一樣。

Here is the new principle with countdown
\begin{code}%
    \>[4]\AgdaPostulate{Finite-State-Termination-With-Countdown} %
    \AgdaSymbol{:} %
    \AgdaSymbol{∀} %
    \AgdaSymbol{\{}\AgdaBound{N} %
    \AgdaBound{st₀}\AgdaSymbol{\}}\<%
    \\
    \>[4][@{}l@{\AgdaIndent{0}}]%
    \>[6]\AgdaSymbol{→} %
    \AgdaField{State} %
    \AgdaOperator{\AgdaFunction{⤖}} %
    \AgdaDatatype{Fin} %
    \AgdaBound{N}\<%
    \\
    %
    \>[6]\AgdaSymbol{→} %
    \AgdaFunction{is-initial} %
    \AgdaBound{st₀}\<%
    \\
    %
    \>[6]\AgdaSymbol{→} %
    \AgdaSymbol{∀} %
    \AgdaBound{cd} %
    \AgdaBound{m} %
    \AgdaBound{stₘ} %
    \AgdaSymbol{→} %
    \AgdaBound{cd} %
    \AgdaOperator{\AgdaPrimitive{+}} %
    \AgdaBound{m} %
    \AgdaOperator{\AgdaDatatype{≡}} %
    \AgdaBound{N} %
    \AgdaSymbol{→} %
    \AgdaBound{st₀} %
    \AgdaOperator{\AgdaDatatype{↦[}} %
    \AgdaBound{m} %
    \AgdaOperator{\AgdaDatatype{]}} %
    \AgdaBound{stₘ}\<%
    \\
    %
    \>[6]\AgdaSymbol{→} %
    \AgdaFunction{∃[} %
    \AgdaBound{stₙ} %
    \AgdaFunction{]} %
    \AgdaSymbol{(}\AgdaBound{st₀} %
    \AgdaOperator{\AgdaDatatype{↦*}} %
    \AgdaBound{stₙ} %
    \AgdaOperator{\AgdaFunction{×}} %
    \AgdaFunction{is-stuck} %
    \AgdaBound{stₙ}\AgdaSymbol{)}\<%
    \\
    \>[0]\<%
\end{code}

\subsection{ The case at N }
Here is the statement of Narrow revTermination after n steps, we have to proof the case is impossible:
\begin{code}%
    \\
    %
    \>[4]\AgdaPostulate{Finite-State-Termination-At-N} %
    \AgdaSymbol{:} %
    \AgdaSymbol{∀} %
    \AgdaSymbol{\{}\AgdaBound{N} %
    \AgdaBound{st₀}\AgdaSymbol{\}}\<%
    \\
    \>[4][@{}l@{\AgdaIndent{0}}]%
    \>[8]\AgdaSymbol{→} %
    \AgdaField{State} %
    \AgdaOperator{\AgdaFunction{⤖}} %
    \AgdaDatatype{Fin} %
    \AgdaBound{N}\<%
    \\
    %
    \>[8]\AgdaSymbol{→} %
    \AgdaFunction{is-initial} %
    \AgdaBound{st₀}\<%
    \\
    %
    \>[8]\AgdaSymbol{→} %
    \AgdaFunction{∃[} %
    \AgdaBound{stₙ} %
    \AgdaFunction{]} %
    \AgdaSymbol{(}\AgdaBound{st₀} %
    \AgdaOperator{\AgdaDatatype{↦[}} %
    \AgdaBound{N} %
    \AgdaOperator{\AgdaDatatype{]}} %
    \AgdaBound{stₙ}\AgdaSymbol{)} %
    \AgdaSymbol{→} %
    \AgdaFunction{⊥}\<%
    \>[0]\<%
\end{code}

The proof of 「當抵達 n-th state 會終止」 can be simply分為 the two steps below:
\begin{enumerate}[1.]
    \item 假定n+1-th state存在, with Piegonhole principle ,可以在 n 個 state 映射到 n + 1 個state,而找到重複經過的兩個相同state
    \item No-repeat 告訴我們找到的兩個state不該是相同的,因為他們的step不同,得到矛盾。
\end{enumerate}.

Before introducing Piegonhole principle and No-Repeat Principle, we will discussing mapping rules.

\subsection{ mapping rules }
// TODO: Fin 相關的說明應該都改為 Index
In our proof of n-th Termination case, a critical part is the mapping rules.
We have,
\begin{enumerate}[1.]
    \item an injection from Step to State.  Note that the 定義域 of step is from 0 to N(included both sides)
    \item bijection between State and Index. because of the limitation of States, we should have n indexes in total.
    \item we can also construct an injection from Step to Index easily.  Just Through the two injection and bijection above.
\end{enumerate}.
Here is an 示意圖 for the relations
\input{ agdaImages/3.3-mapping-relation }

Note again that there are only n indexes in the relations.
With the relation, we can 更清楚地描述 the Termination proof at n-th state:
\begin{enumerate}[1.]
    \item first of all, we have n+1 states (from $st_{0}$ to $st_{n}$)
    \item with injection from Step to index, we get two different steps mapping to same index using Piegonhole Principle
    \item with bijection from index back to State, the same indexes should be mapped to same states.
    \item No-repeat principle told us that the two different steps should be mapped to different states, and the contradiction occurs.
\end{enumerate}.

\subsection{ Piegonhole principle }
The definition of Piegonhole principle in agda is shown below:

\input{ agdaCode/3.3-piegonhole-principle }

The two necessary inputs are 
\begin{enumerate}[1.]
    \item N to N function. In this case, it is the injection from Step to Index. 
    \item all mapped N should be less than N.  The proof is 隱含在 Fin的定義裡面.  That is, a natrual number value of Fin N is always less than N.
\end{enumerate}.

And we could get two different steps which will injection to the same index.

\subsection{ No Repeat }
The definition of No-Repeat-Principle in agda is shown below:

No-Repeat-Principle is proved by {TODO: 學長論文}.  In the principle, we know that the two states with different steps from initial state, they should be different states.
\begin{code}%
    \\
    %
    \>[4]\AgdaPostulate{NoRepeat} %
    \AgdaSymbol{:} %
    \AgdaSymbol{∀} %
    \AgdaSymbol{\{}\AgdaBound{st₀} %
    \AgdaBound{stₙ} %
    \AgdaBound{stₘ} %
    \AgdaBound{n} %
    \AgdaBound{m}\AgdaSymbol{\}}\<%
    \\
    \>[4][@{}l@{\AgdaIndent{0}}]%
    \>[10]\AgdaSymbol{→} %
    \AgdaFunction{is-initial} %
    \AgdaBound{st₀}\<%
    \\
    %
    \>[10]\AgdaSymbol{→} %
    \AgdaBound{n} %
    \AgdaOperator{\AgdaFunction{<}} %
    \AgdaBound{m}\<%
    \\
    %
    \>[10]\AgdaSymbol{→} %
    \AgdaBound{st₀} %
    \AgdaOperator{\AgdaDatatype{↦[}} %
    \AgdaBound{n} %
    \AgdaOperator{\AgdaDatatype{]}} %
    \AgdaBound{stₙ}\<%
    \\
    %
    \>[10]\AgdaSymbol{→} %
    \AgdaBound{st₀} %
    \AgdaOperator{\AgdaDatatype{↦[}} %
    \AgdaBound{m} %
    \AgdaOperator{\AgdaDatatype{]}} %
    \AgdaBound{stₘ}\<%
    \\
    %
    \>[10]\AgdaSymbol{→} %
    \AgdaBound{stₙ} %
    \AgdaOperator{\AgdaFunction{≢}}%
    \>[19]\AgdaBound{stₘ}\<%
    \\
    \>[0]\<%
\end{code}

Until now, we have 補足了 the 剩餘的部分 of Narrow revTermination Principle, 從而補足了整體的證明。