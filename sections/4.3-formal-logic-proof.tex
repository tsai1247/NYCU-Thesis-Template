\section{ Formal Logic Proof }

Base on the informal logic proof, we continue 討論 the descriptions of the formal one.
Same as the narrow reversible termination proof, we have the following steps:
\begin{enumerate}[1.]
    \item Starting from initial state, try looking for its next state. 
    \item If the next state exists, keep looking for 再下一個 state
    \item Upon reaching the n-th state, proof that it's impossible for the existence of next states with no-repeat principle.
\end{enumerate}.


\subsection{ Countdown rules }
We 引入 countdown rule to make agda terminate.

\input{agdaCode/4.3-broad-revTermination-with-countdown}
Our target Principle can be regarded as the special case when Countdown is N, 也就是剛開始的時候。
With countdown rule, we could check if each state we through has next state.  Once it doesn't have next state, it will terminate immediately.  Otherwise, we will get N-th state.

\subsection{ At n-th state }
The proof of 「當抵達 n-th state 會終止」 can be simply分為 the two steps below:
\begin{enumerate}[1.]
    \item 假定n+1-th state存在, with Piegonhole principle ,可以在 n 個 state 映射到 n + 1 個state,而找到重複經過的兩個相同state
    \item No-repeat 告訴我們找到的兩個state不該是相同的,因為他們的step不同,得到矛盾。
\end{enumerate}

Here is the mapping graph to explain the steps:
\vspace{1em}

\usetikzlibrary{graphs, positioning, quotes, shapes.geometric}

\begin{document}
\begin{tikzpicture}[node distance=10pt]
    \node[draw]                         (Step 0)  {Step 0};
    \node[draw, below=30pt of Step 0]   (Step 1)  {Step 1};
    \node[draw, below=30pt of Step 1]   (Step 2)  {Step 2};
    \node[draw, below=30pt of Step 2]   (Step 3)  {Step 3};
    \node[draw, below=30pt of Step 3]   (Step n)  {Step n};
    
    \node[draw, right=30pt of Step 0]   (State 0)  {State 0};
    \node[draw, right=30pt of Step 1]   (State 1)  {State 1};
    \node[draw, right=30pt of Step 2]   (State 2)  {State 2};
    \node[draw, right=30pt of Step 3]   (State 3)  {State 3};
    \node[draw, right=30pt of Step n]   (State n)  {State n};

    \node[draw, right=30pt of State 0]   (Index 0)  {Index 0};
    \node[draw, right=30pt of State 1]   (Index 1)  {Index 1};
    \node[draw, right=30pt of State 2]   (Index 2)  {Index 2};
    \node[draw, right=30pt of State 3]   (Index 3)  {Index 3};
    \node[draw, right=30pt of State n]   (Index n)  {  ?  };

    \graph{
        (Step 0) -> (State 0) -> (Index 1);
        (Step 1) -> (State 1) -> (Index 3);
        (Step 2) -> (State 2) -> (Index 0);
        (Step 3) -> (State 3) -> (Index 2);
        (Step n) -> (State n) -> (Index n);
    };
\end{tikzpicture}

The number of reachable state is only n.  That is, the Step n should be mapping to 重複的 index.  And we'll 藉由 the bijection to find out state n is also 重複 with one of previous states.
It's a contradiction with No-repeat principle.

