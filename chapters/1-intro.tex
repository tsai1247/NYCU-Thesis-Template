\chapter{Introduction}
\label{chapter:intro}

\section{Motivation}

In CHAO-HONG CHEN's study~\cite[1], three significant contributions are meticulously elaborated. 
Initially, the research formalizes the concepts of "reversing time" and "reversing space," showcased through the compact closed categories (N,+, 0) and (N, *, 1) respectively; 
secondly, it further implements a reversible SAT solver, proving the feasibility of high-level control abstractions described in the first contribution within reversible programming languages; 
the third contribution focuses on proving the termination conditions for a large class of reversible abstract machines.

Specifically, the paper presents two combinatorial proofs in Chapters 5 and 7, regarding the property that a class of reversible abstract machines will inevitably terminate for initial states with a finite number of reachable states. 
This proof, although intuitively obvious — that is, on a path of unique states, the number of reachable states decreases as the state progresses, until no further states can be reached, or a termination state is achieved. 
However, the original paper only partially formalized this proof, failing to fully substantiate this concept.

Our contribution starts from constructing a similar class of reversible abstract machines. 
Begin with a given initial state, demonstrating that when all possible states of a reversible abstract machine are finite, it will ultimately reach a termination state.
Subsequently, we modified the constraints to align with the objective of "finite reachable states" mentioned in the original paper, thereby accomplishing a more generalized proof.

This paper is accompanied by approximately 400 lines of Agda code, serving to complement the proofs of the two theorems mentioned in the original paper as being incomplete.

\section{Research Objectives}


\section{Road Map}
TODO
Chapter~\ref{chapter:doccls} introduces thesis.cls document class.
Chapter~\ref{chapter:secorder} introduces section ordering.
Chapter~\ref{chapter:ref} and~\ref{chapter:fig} explain how to load citation, figures and tables (not completed).