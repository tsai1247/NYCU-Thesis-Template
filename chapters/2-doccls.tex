\chapter{Background Knowledge}
\label{chapter:doccls}

In this chapter, we introduce the target language used in the research: Agda. 
As a dependently typed functional programming language, we provide a foundational explanation of its usage.

\section{Agda}

Agda is a dependently typed functional programming language [3].
Writing in the Agda language is akin to organizing a mathematical proof. It begins with self-evident datatypes, proposes assumptions to be proven, and provides detailed evidence for their validity thereafter.
The following is an example of a datatype concerning natural numbers[5]. 

\begin{code}%
    \>[0]\<%
    \\
    \>[0]\AgdaKeyword{data} %
    \AgdaDatatype{ℕ} %
    \AgdaSymbol{:} %
    \AgdaPrimitive{Set} %
    \AgdaKeyword{where}\<%
    \\
    \>[0][@{}l@{\AgdaIndent{0}}]%
    \>[2]\AgdaInductiveConstructor{zero}%
    \>[8]\AgdaSymbol{:} %
    \AgdaDatatype{ℕ}\<%
    \\
    %
    \>[2]\AgdaInductiveConstructor{suc}%
    \>[8]\AgdaSymbol{:} %
    \AgdaDatatype{ℕ} %
    \AgdaSymbol{→} %
    \AgdaDatatype{ℕ}\<%
    \\
    \>[0]\<%
\end{code}

This definition captures the first two axioms corresponding to Peano's axioms, and the other axioms within Peano's axioms can also be easily proven. 
The function suc can be considered as a mapping operation. 
We can describe this operation in natural language as follows: 
"Given a natural number n, suc(n) is also a natural number." 
Such a function from $\mathbb{N}$ to $\mathbb{N}$ will be utilized in Chapter {TODO}, where it will constitute a critical part of the proof process.

And the following demonstrates two examples of the proof process:

\begin{code}%
\>[0]\AgdaFunction{axioms-3} %
\AgdaSymbol{:} %
\AgdaFunction{∄[} %
\AgdaBound{n} %
\AgdaFunction{]} %
\AgdaSymbol{(}\AgdaInductiveConstructor{suc} %
\AgdaBound{n} %
\AgdaOperator{\AgdaDatatype{≡}} %
\AgdaInductiveConstructor{zero}\AgdaSymbol{)}\<%
\\
\>[0]\AgdaFunction{axioms-3} %
\AgdaSymbol{()}\<%
\\
\>[0]\<%
\end{code}

Here, we take the third Peano axiom as an example, which is to prove that there does not exist a natural number such that suc(n) is zero. 
In Agda, a proof of non-existence typically begins by assuming the existence of such an entity, and then deriving a contradiction (denoted as $\bot$). 
In this case, the proof process is merely an empty parenthesis.
When Agda attempts to find n among the two possible kinds of natural numbers (zero and suc(n)), it immediately encounters an obvious contradiction. 
Therefore, it concludes that no further cases need to be proven.

\begin{code}%
\>[0]\AgdaFunction{axioms-5} %
\AgdaSymbol{:} %
\AgdaSymbol{∀} %
\AgdaSymbol{\{}\AgdaBound{f} %
\AgdaSymbol{:} %
\AgdaDatatype{ℕ} %
\AgdaSymbol{→} %
\AgdaPrimitive{Set}\AgdaSymbol{\}}\<%
\\
\>[0][@{}l@{\AgdaIndent{0}}]%
\>[2]\AgdaSymbol{→} %
\AgdaBound{f} %
\AgdaNumber{0}\<%
\\
%
\>[2]\AgdaSymbol{→} %
\AgdaSymbol{(∀} %
\AgdaBound{n} %
\AgdaSymbol{→} %
\AgdaBound{f} %
\AgdaBound{n} %
\AgdaSymbol{→} %
\AgdaBound{f} %
\AgdaSymbol{(}\AgdaInductiveConstructor{suc} %
\AgdaBound{n}\AgdaSymbol{))}\<%
\\
%
\>[2]\AgdaSymbol{→} %
\AgdaSymbol{(∀} %
\AgdaBound{n} %
\AgdaSymbol{→} %
\AgdaBound{f} %
\AgdaBound{n}\AgdaSymbol{)}\<%
\\
\>[0]\<%
\end{code}
This statement corresponds to the fifth Peano axiom. 
Agda uses the right arrow to sequentially assign the necessary conditions, with the conclusion appearing to the right of the last right arrow. 
The fifth Peano axiom describes that for any function f, if:
\begin{enumerate}[1.]
\item f(0) holds true,
\item f(n) being true implies that f(suc(n)) is also true
\end{enumerate}
then for all natural numbers n, f(n) holds true.

\begin{code}%
    \>[0]\AgdaFunction{axioms-5} %
    \AgdaBound{f0} %
    \AgdaBound{fn-sucn} %
    \AgdaInductiveConstructor{zero} %
    \AgdaSymbol{=} %
    \AgdaBound{f0}\<%
    \\
    \>[0]\AgdaFunction{axioms-5} %
    \AgdaBound{f0} %
    \AgdaBound{fn-sucn} %
    \AgdaSymbol{(}\AgdaInductiveConstructor{suc} %
    \AgdaBound{n}\AgdaSymbol{)}\<%
    \\
    \>[0][@{}l@{\AgdaIndent{0}}]%
    \>[2]\AgdaKeyword{with} %
    \AgdaFunction{axioms-5} %
    \AgdaBound{f0} %
    \AgdaBound{fn-sucn} %
    \AgdaBound{n}\<%
    \\
    \>[0]\AgdaSymbol{...} %
    \AgdaSymbol{|} %
    \AgdaBound{fn} %
    \AgdaSymbol{=} %
    \AgdaBound{fn-sucn} %
    \AgdaBound{n} %
    \AgdaBound{fn}\<%
    \\
    \>[0]\<%
\end{code}
This illustrates the proof of the fifth Peano axiom. This straightforward proof highlights a few commonly used proof techniques in Agda:
\begin{enumerate}[1.]
\item Pattern Matching on Variables: The example splits the natural number n into two cases: zero and suc n, for separate discussion. This technique allows for detailed examination of different scenarios directly related to the structure of natural numbers.
\item Well-founded recursion in Agda [6]: Theorems involving natural numbers often leverage induction. In Agda, we typically prove a case for n to infer the case for suc n. Agda ensures that there is a corresponding proof for the base case (usually zero), which acts as the termination condition for a comprehensive proof.
\item Simplifying Variables Using 'with': In the example, the proof for f(n) is obtained using axiom-5, and it is named fn. This approach helps to avoid verbose variable expressions, streamlining the proof.
\end{enumerate}

\section{Reversible Abstract Machine}

The Reversible Abstract Machine, abbreviated as RevMachine, is defined as follows.

\begin{code}%
\>[0][@{}l@{\AgdaIndent{0}}]%
\>[2]\AgdaKeyword{field}\<%
\\
\>[2][@{}l@{\AgdaIndent{0}}]%
\>[4]\AgdaField{State} %
\AgdaSymbol{:} %
\AgdaPrimitive{Set} %
\AgdaBound{ℓ}\<%
\\
%
\>[4]\AgdaOperator{\underline{\hspace{0.5em}}\AgdaField{↦}\underline{\hspace{0.5em}}} %
\AgdaSymbol{:} %
\AgdaFunction{Rel} %
\AgdaField{State} %
\AgdaBound{ℓ}\<%
\\
%
\>[4]\AgdaField{deterministic} %
\AgdaSymbol{:} %
\AgdaSymbol{∀} %
\AgdaSymbol{\{}\AgdaBound{st} %
\AgdaBound{st₁} %
\AgdaBound{st₂}\AgdaSymbol{\}} %
\AgdaSymbol{→} %
\AgdaBound{st} %
\AgdaOperator{\AgdaField{↦}} %
\AgdaBound{st₁} %
\AgdaSymbol{→} %
\AgdaBound{st} %
\AgdaOperator{\AgdaField{↦}} %
\AgdaBound{st₂} %
\AgdaSymbol{→} %
\AgdaBound{st₁} %
\AgdaOperator{\AgdaDatatype{≡}} %
\AgdaBound{st₂}\<%
\\
%
\>[4]\AgdaField{deterministicᵣₑᵥ} %
\AgdaSymbol{:} %
\AgdaSymbol{∀} %
\AgdaSymbol{\{}\AgdaBound{st} %
\AgdaBound{st₁} %
\AgdaBound{st₂}\AgdaSymbol{\}} %
\AgdaSymbol{→} %
\AgdaBound{st₁} %
\AgdaOperator{\AgdaField{↦}} %
\AgdaBound{st} %
\AgdaSymbol{→} %
\AgdaBound{st₂} %
\AgdaOperator{\AgdaField{↦}} %
\AgdaBound{st} %
\AgdaSymbol{→} %
\AgdaBound{st₁} %
\AgdaOperator{\AgdaDatatype{≡}} %
\AgdaBound{st₂}\<%
\\
\>[0]\<%
\end{code}

"State" is the set of all states.

"$\mapsto$" is used to record state transitions, for example, $st_{0}$ $\mapsto$ $st_{1}$ indicates that $st_{0}$ transitions to $st_{1}$.

"Deterministic" refers to forward determinism, meaning that identical states will transition to the same next state.

"deterministicᵣₑᵥ", on the other hand, is the opposite of "deterministic", indicating that for each state, all possible transitions to it are the same.

Based on the definitions above, given a RevMachine and one of its states, we can determine an invariant trace composed of states.

