\begin{abstract}%
    In Chao-Hong Chen's paper, ``A Computational Interpretation of Compact Closed Categories: Reversible Programming with Negative and Fractional Types,'' he explores an interpretation of compact closed categories through ``reversing time and space,'' formalizing its application in the development of a reversible SAT solver. 
    The termination proof of a reversible abstract machine, an essential component of the paper, involved the formalization of two crucial theorems.
    Although these proofs are clearly correct, only partial formal proofs are provided in the paper.

    Our research delves deeply into this statement from Chen's work. Initially, we rearticulate the specifics of the statement: for a reversible abstract machine, given an initial state with a known finite number of reachable states, this initial state will terminate within a finite number of transitions through the reversible abstract machine. 
    Subsequently, we added a condition and completed a formal proof under the assumption that all states are finite.
    Finally, we removed the assumption of finite states and employed similar methods to complete a formal proof under infinite states, thus complementing and enhancing the proofs in the original paper.
    
\vspace{17cm}

Keyword: Abstract Machines, Termination Proofs, Agda

\end{abstract}