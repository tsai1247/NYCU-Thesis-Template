\begin{abstractzh}
    在 陳昭宏 (Chao-Hong Chen) 的論文《A Computational Interpretation of Compact Closed Categories: Reversible Programming with Negative and Fractional Types》中,他探討了緊湊封閉類別透過「反轉時間」與「反轉空間」的解釋,並將其形式化應用於可逆SAT求解器的開發上。
    該論文中,對可逆抽象機的終止證明尤為關鍵,涉及兩項重要定理的形式化證明。儘管這些證明顯然正確,在該論文中尚僅提供了部分的形式化證明。
    
    我們的研究深入探討了論文中的這一陳述。
    首先,明確重申了陳述的具體內容:對於可逆抽象機,給予一初始狀態,並且已知該初始狀態可抵達的狀態是有限的,則該初始狀態經由可逆抽象機的推移,將會在有限的步數內終止。
    隨後,我們為此定理附加了一項條件並完成了相關的形式化證明,即所有狀態的個數有限的情況下,對該陳述的形式化證明。
    最後,我們移除了有限狀態的假設,並以類似方法完成了無限狀態下的形式化證明,從而補充並完善了論文中該部分的證明。

\vspace{17cm}

關鍵詞: 可逆抽象機、終止證明、Agda

\end{abstractzh}